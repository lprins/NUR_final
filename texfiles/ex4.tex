\section*{Exercise 4}
We use these constants in this exercise:

\inputminted[firstline=12, lastline=24]{Python}{../ex4.py}

\paragraph{4a}
We want to calculate
\begin{equation}
    D(z) = \frac{5 \Omega_m H_0^2}{2} H(z)
    \int_z^\infty \frac{1 + z'}{H^3(z')} dz'
\end{equation}
Note that
\begin{equation*}
    a = \frac{1}{1+z} \implies dz = -\frac{da}{a^2}
\end{equation*}
So that this is equivalent to
\begin{equation}
    D(a) = \frac{5 \Omega_m }{2} \frac{H(a)}{H_0}
    \int_0^a {\left(a' \frac{H(a')}{H_0}\right)}^{-3} da'
\end{equation}
We know from the Friedmann equation
\begin{equation}
    \frac{H(a)}{H_0} = {(\Omega_m a^{-3} + \Omega_\Lambda)}^{1/2}
\end{equation}
So we can calculate $D(z=50) = D(a=1/51)$ by numerical integration.
We need to perform an open integration, because there is a singularity at $a' = 0$.
For this we use the open Romberg integration routine shown below.

\inputminted[firstline=37, lastline=57]{Python}{../integration.py}
\inputminted[firstline=68, lastline=80]{Python}{../ex4.py}
\inputminted[firstline=146, lastline=149]{Python}{../ex4.py}

With the result shown below.

\inputminted[firstline=1, lastline=1]{text}{../output/ex4.txt}

To calculate $\dot{D}$, we use that
\begin{equation*}
    \dot{D} = \frac{dD}{da} \dot{a} = \frac{dD}{da} a H
\end{equation*}
and derive analytically that
\begin{equation*}
    \frac{dD}{da} = \frac{5 \Omega_m}{2 a^3} {\left(\frac{H}{H_0}\right)}^{-1}
    \left( {\left(\frac{H}{H_0}\right)}^{-1} -
    \frac{3 \Omega_m}{2a} \int_0^a {\left(a' \frac{H(a')}{H_0}\right)}^{-3} da' \right)
\end{equation*}
so that
\begin{equation}
    \dot{D} = \frac{5 H_0 \Omega_m}{2 a^2}
    \left( {\left(\frac{H}{H_0}\right)}^{-1} -
    \frac{3 \Omega_m}{2a} \int_0^a {\left(a' \frac{H(a')}{H_0}\right)}^{-3} da' \right)
\end{equation}
Where we use the numerical result for the integral we obtained earlier.

We can also calculate the derivative $\frac{dD}{da}$ numerically and then multiply by $\dot{a}$.
We use Ridders' method to do this.

\inputminted[firstline=73]{Python}{../integration.py}
\inputminted[firstline=82, lastline=86]{Python}{../ex4.py}
\inputminted[firstline=151, lastline=158]{Python}{../ex4.py}

With results shown below
\inputminted[firstline=2]{text}{../output/ex4.txt}
As we can see, the agreement between the analytical and the numeric derivative is very good.

\paragraph{4c}
Using the Zeldovich approximation, we can generate initial conditions in 2D.
We use the 2D IFFT from numpy to generate initial conditions for $64^2$ particles starting from $z=50$, and enforce the periodic boundary conditions using the mod (\%) operator.
We also use a utility function to calculate the frequencies in the Fourier plane.

\inputminted[firstline=74]{Python}{../fft.py}
\inputminted[firstline=88, lastline=111]{Python}{../ex4.py}
\inputminted[firstline=160, lastline=207]{Python}{../ex4.py}

A movie of the evolution can be found in \texttt{movies/2D\_zeldovich/ICs.mp4}

\begin{figure}[h]
  \centering
  \includegraphics[width=0.9\linewidth]{ex4_2D_ypos}
  \caption{y-positions of the first 10 particles in the 2D ZA as a function of scale factor.}
\end{figure}

\begin{figure}[h]
  \centering
  \includegraphics[width=0.9\linewidth]{ex4_2D_yvel}
  \caption{y-velocities of the first 10 particles in the 2D ZA as a function of scale factor.}
\end{figure}

\clearpage

\paragraph{4d}
We can also do this in 3D.
We generate initial conditions for a later simulation, and plot the evolution of the initial conditions over time.
Movies can be found in \texttt{movies/3D\_zeldovich/xy/ICs.mp4}, \texttt{movies/3D\_zeldovich/xz/ICs.mp4}, \texttt{movies/yz/3D\_zeldovich/ICs.mp4} for slices in the xy, xz and yz plane respectively.

\inputminted[firstline=113, lastline=144]{Python}{../ex4.py}
\inputminted[firstline=209]{Python}{../ex4.py}

\begin{figure}[h]
  \centering
  \includegraphics[width=0.9\linewidth]{ex4_3D_zpos}
  \caption{z-positions of the first 10 particles in the 3D ZA as a function of scale factor.}
\end{figure}

\begin{figure}[h]
  \centering
  \includegraphics[width=0.9\linewidth]{ex4_3D_zvel}
  \caption{z-velocities of the first 10 particles in the 3D ZA as a function of scale factor.}
\end{figure}
\clearpage
